\documentclass{article}
\usepackage{graphicx} % Required for inserting images
\usepackage{subcaption}
\usepackage{amsmath}
\usepackage{pdflscape}

\author{José Carlos González Martínez}
\date{May 2024}

\begin{document}

\section*{Massive Black Holes Theory}

\subsection*{Massive Black Holes in Galactic Nuclei}

Structure formation encompasses the description of cosmic structures, ranging from galaxies, up to clusters of galaxies, to large scale filaments. 
Large scale structures formed at the time of inflation, from the gravitational growth of initial matter inhomogeneities. The evolution of these structures depends on the initial power spectrum of matter inhomogeneities, which have been now measured, and on the expansion of the universe, which depends on the Universe's content of dark matter, baryonic matter and dark energy. 

Much of the knowledge of cosmic structure that we have comes from cosmological simulations, where a vast range of space and time scales are involved: On the one hand, gravitational effects of super clusters of galaxies needs to be accounted for, and on the other hand, these effects have impact on the gas that cools down to form stars. Ultimately, large scale structure formation should give us insights on how galaxies form and evolve, so that we can tackle deeper problems such as the nature of dark matter and dark energy.  

In this area, there are challenges that need to be worked out: The simulations need to cover a significant volume in order to represent the Universe in a statistically correct manner, but at the same time, they need to have the sufficient resolution to model the physics at galactic and sub-galactic scales. At present times, cosmological simulations have relied on models based on dark matter-only, missing out the part that astronomers actually observe, i.e. baryonic, ordinary matter. On the other hand, galaxy formation and evolution has a close connection with massive black holes, in that massive black holes are hosted at the center of most galaxies. Particularly, in hydrodynamic cosmological simulations, phenomena such as black hole physics, star formation, gravity and hydrodynamics are taken into account, so modelling at large and small scales is coupled consistently. In recent simulations the physics included is 1) Star formation and ISM modelling, 2)Gas-cooling and photoionisation, and 3) Stellar evolution and feedback.

\subsection*{Massive Binary Black holes in Galactic Nuclei}

Black Holes exist in two categories: Stellar black holes, with masses of 5 to 30 Msun, which form from the collapse of stars that undergo supernovae explosion. These can be detected through their X-Ray emission. On the other hand, there exist massive black holes with masses in the range of 10⁵ to 10⁹ which power QSOs and active galactic nuclei (AGN), and have been observed in nearby galaxy spheroids and certainly, our Milky Way galaxy host an almost inactive massive black hole. There is a gap between these two categories, called intermediate mass or middleweight black holes which has not been accurately constrained due to the lack of observations. Stellar black holes, on the other hand, can have masses as large as 10² Msun depending on the metallicity of the progenitor stars, and on radiation feedback.
Through X-Ray observations, there is evidence that massive black holes have grown in mass through episodes of mergers and coalescences driven by galaxy mergers. This has led to the concept of black hole seeds, which masses are theoretically weakly constrained to this day.
The discovery of the correlations between the black hole mass and the stellar velocity dispersion, and the black hole mass and the stellar mass of the spheroid, indicate a coupled evolution between the massive black hole and its host galaxy

At present, there is a debate in which this interpretation holds true to bulge-less disc galaxies, or if it holds in general to lower-mass galaxies (reference). If the latter case is true, intermediate mass black holes are could be hosted in low-mass galaxies (reference). In 75(percent) of the observed galaxies, there is a Nuclear Star Cluster inhabiting their center. It has been observed that intermediate mass black holes inhabit there Nucelar Star Clusters. As another mean to explore this possibility, space-based gravitational wave observatory LISA is expected to search and detect low frequency gravitational waves from binary black holes from merging galaxies, thus exploring the coevolution of massive black holes and its host galaxy. 

According to the LCDM paradigm, galaxies form after the infall of baryonic matter into dark matter halos, leading to the formation of black hole seeds, which grow as a consequence of multiple mergers and coalescences. As part of the goal mission, LISA will be able to explore the formation and evolution of these black hole seed and by constraining the masses as early as z=10 (reference). The study of the formation and evolution of binary black holes inside galactic halos as the largest cosmic structures evole crosses boundaries between astrophysics and cosmology.

\subsection*{Binary Black Holes in Stellar Environments}

Begelman et al. (1980) listed three stages of the coalescence of two massive black holes in a galaxy merger: 1) Pairing, which is when the black holes orbit shrinks due to dynamical friction, 2) Hardening, in which the orbit further decreases due to energy lose by close encounters with sinlge stars, and 3) gravitational wave emission. After coalescence, there is a gravitational recoil upon the merged black holes which can be as large as 5000 km s-1.

\subsection*{Populations of MBHBs}

MBHBs are ubiquitous in the universe and are thought to be at the center of a large fraction of galaxies. Up to date, our knowledge from MBHBs have come from electromagnetic observations, cosmological simulations, and theoretical tools, but there are still missing some pieces of the full puzzle regarding their formation and evolution. From the aggregation of galaxies through merger episodes, a single merger remnant is formed hosting two MBHs. The MBHs start a closer interaction formaing a bound binary. When the distance scale between them decrese below the kpc scale, they form a binary where gravitational wave emission takes place. 

One of the fundamental questions that arise from the picture described above is about MBHBs formation and mass acrettion through cosmic history. Current MBHB formation and evolution theories suggest that MBHBs, apart from primordial MBHBs, are formed from light seeds and heavy seeds. On the one hand, light seeds are composed from population III, metal-free star remnants. On the other hand, havey seeds are formed  from the collapse of super massive stars or by the dynamic interaction of stellar mass clusters, respectively (reference). Observations of unusually massive quasars at high redshift suggest that MBHBs have their origins largely from heavy seeds or at least dominate it (reference).

Another question regarding the link of the formation of MBHBs with galaxy mergers is how two MBHs at kpc distances inside a galaxy merger remnant get closer so that gravitational wave emission dominates the evolution of the binary. Three evolutionay stages have been proposed in this regard (reference): 1) Pairing, 2) Hardening, 3) Gravitational wave emission. During the first stage, pairing, the interaction of the two MBHs with the background of stars, gas and dark matter is dominated by dynamical frcition which generates a drag force, shrinking the distance between the two MBHs to a point where a binary is formed. During the second stage, hardening,...(reference MBHBs galactic nuclei theory papers) the MBH binary (MBHB) loses energy and anuglar momentum due to three-body encounters with stars in the surroundings shrinking the orbit to pc scales. At this point the final parsec problem arises reflecting the challenges involved in understanding how the MBHB reaches closer distance scales where gravitational wave emission dominates the interaction (reference). The final stage  corresponds to gravitational wave emission. Here, the space-based GW observatory LISA will observe MBHBs with masses between 10⁴ and 10⁷ Msun up to z~20 with an expected observation rate from a few to a few hundreds of MBHBs per year (reference)...(expand with more papers)

POPULATIONS (?)

MBHB population studies allow us to know how MBHBs form and evolve, and what is the relation with the host galaxy and their environments. This is achieved by relating the physical mechanisms that are involved in the merging process to the GW observations of MBHBs. To connect these parts, MBHB population models are used, and they describe the distribution of the parameters of the MBHBs in the Universe along with the characterisation of MBHB systems with GW data analysis tools. MBHB population models are constructed, on the one hand, via cosmological simulations, where MBHs, galaxies, clusters of galaxies and dark matter are at play. Onth other hand, there are analytic and semi-analytic models which rely of physical assumptions about the mechanisms involved in the mergers.

\section*{Observations of MBHB with LISA and GWFish}

\subsection*{GWFish}

\subsubsection*{Overview}
There exist a whole network of GW ground-based observatories: LIGO, Virgo, KAGRA, which cover the same range of GW sources but also with capabilities of observing different frequency bands by combining observations of the same GW sources. The advanced ground-based detectors like Einstein telescope and Cosmic Explorer will extend the observational capabilities of this ground-based network. Proposed space-based observatory LISA on the other hand will cover a brand new territory in the GW spectrum. Apart from developing detection infrastructure, the GW comunity has put effort in developing computational tools to explore the capabilities of the current and future detectors. One of these tools consist in building software capable of simulate GW observations, however, there are challenges when tackling this problem. 

GWFish is a simulation software intended for estimating parameter estimation uncertainties. It makes use of Fisher matrices in the high SNR regime and Gaussian likelihood approximation. This is performed with time domain GW models and frequency domain of detector network's  response. With this framework, it is possible to carry out parameter estimation studies where the position and orientation of the network of detectors changes with time, which has an important impact mainly in sky localization.

GWFish can perform multiband simulated observations as well. This is possible since Fisher matrix uncertainties from different detectors can be added to the overall estimation irrespective of the frequency band that provided the signal information. This also applies to the time delay interferometry (TDI) used by detectors like LISA.

The main challenges when estimating parameter estimation uncertainties involve the computation of the waveform derivatives, and the inversion of the Fisher matrices. GWFish takes an hybrid approach for the waveform derivatives with both analytical and numerical differentiation, the latter tuned up to the waveform parameters to reduce numerical errors. Regarding the inversion of the Fisher matrices, in general, the main issue is that Fisher matrices are close to singular, meaning that the Fisher matrices eigenvalues span a huge range, leading to signal-model degeneracies. To be able to estimate parameter uncertainties, at least of the parameters not involved in the degeneracy, one needs to deal with matrix singularity by means of techniques like Singular Value Decomposition (SVD). 

\subsubsection*{Waveform analytical and numerical derivatives.}

GWFish computes the derivatives of the waveform in a hybrid analytical-numerical fashion. The analytical derivatives of the waveform $\Tilde{h}_{k}(\theta^{j},f)$ with respect to the waveform phase $\varphi_{c}$, the luminosity distance $d_{L}$ and merger time $t_{m}$ are,

\begin{equation}
\frac{\partial \Tilde{h}_{k}(\theta^{j},f)}{\partial \varphi_{c}} = -i \Tilde{h}_{k}(\theta^{j},f),
\label{phasederiv}
\end{equation}
\begin{equation}
\frac{\partial \Tilde{h}_{k}(\theta^{j},f)}{\partial d_{L}} = -\frac{\Tilde{h}_{k}(\theta^{j},f)}{d_{L}},
\label{lumdistderiv}
\end{equation}
\begin{equation}
\frac{\partial \Tilde{h}_{k}(\theta^{j},f)}{\partial t_{m}} = 2\pi i f \Tilde{h}_{k}(\theta^{j},f),
\label{timecderiv}
\end{equation}

The derivatives for the other parameters (component masses, spins, sky localisation, inclination, polarisation angle) are computed numerically using the expression,
\begin{equation}
\frac{\partial \Tilde{h}_{k}(\theta^{j},f)}{\partial \theta^{i}} \approx \frac{\Tilde{h}_{k}(\theta^{j} +\epsilon^{ij}/2,f) - (\theta^{j} -\epsilon^{ij}/2,f)}{\epsilon},
\label{numderiv}
\end{equation}
where $\epsilon$ is the step size of the derivatives.
\subsubsection*{Fisher matrix inversion. Singular Value Decomposition.}
Given that the Fisher matrices we are dealing with are close to singular, i.e. matrices with small or zero eigenvalues, its inversion is prone to numerical instabilities or inaccurate estimations. For these type of cases the Singular Value Decomposition (SVD) technique is used. It ensures that the inversion of the Fisher matrix is performed free of eigenvalues close to zero. In order to apply SVD, the original matrix is normalized by dividing it by the outer product of the root-squared elements of the diagonalised matrix. Then SVD is applied following the expression,

\begin{equation}
A = U S V^{h},
\label{SVD}
\end{equation}
where $A$ is the original matrix, $U$ and $V^{h}$ are orthogonal matrices, and $S$ is the matrix with the singular values. Threshold value for any entry in $S$ is set to $1 \times 10^{-10}$, so values below this threshold are ignored. Then the matrix is inverted in a process called pseudo-inversion as this avoids values below the threshold. The remaining matrix is then denormalised by multiplying back by the normalisation factor to provide a new Fisher matrix without values close to zero.  

\subsubsection*{Detector-Network Simulation}
There are several aspects that GWFish takes into account when simulating the network of detectors:
\begin{itemize}
    \item Component. For the case of LISA, the component is the TDI interferometry and includes the noise model, the duty cycle and  
    \item Detector. This includes the detector response.
    \item Network.
\end{itemize}
In the detector response, the position of the detector and its orientation needs to be considered. This is calculated with the expression,
\begin{equation}
\mathcal{R}(f) = \mathcal{A}(t(f)) : h(f),
\label{responseeq}
\end{equation}
where $\mathcal{R}(f)$ is the response function, $\mathcal{A}(t(f))$ is the response tensor and $t(f)$ maps signal frequencies to times, and is derived from the phase of the signal $\phi (f)$ through,
\begin{equation}
t(f) = \frac{1}{2\pi} \frac{d \phi (f)}{d f}.
\label{ftmap}
\end{equation}
In general, the early inspiral part of a binary system is the most important part for the detector motion simulation, so $\phi (f)$ can be approximated to lowest order.
GWFish simulates LISA's response by considering the TDI interferometry; in this technique, the main building blocks are the readouts $y_{ij}$ of every spacecraft link $j \longleftarrow i$, and by applying a time delay, it reduces laser noise. Indeed, every spacecraft acts as the vertix of a triangular network where each of them is a laser interferometer. Since the noise is correlated, the noise-correlated matrix needs to be diagonalised. This leads to three separate channels: A, E, and T channels. GWFish simulates these three channels and also applies the breathing motion approximation where the fluctuations in the arm lengths of the detector over the curse of a year are neglected.

\subsubsection*{Setup}
GWFish requires setting up a range of parameters which describe the systems to be simulated, the response of the detector, the network of detectors, and details regarding the GW signal such as the waveform approximant, the SNR threshold... 


\subsection*{Transformation of variables}

When estimating the pdf $p(X,Y|I)$ where $X$ and $Y$ are the constrained parameters and $I$ is the background information, we might be interested in the posterior $p(Z|I)$, where $Z$ is derived from $X$ and $Y$, $Z=f(X,Y)$. This what error-propagation is about, and is performed by means of variable transformation (Sivia). In the case of one variable, we might ask how $p(X|I)$ is related to $p(Y|I)$ if $Y=f(X)$?. Let's say that $\delta X$ is a very small interval about $X=X^{\ast}$, the probability that $X$ lies in the range between $X^{\ast}-\delta X/2$ and $X^{\ast} + \delta X/2$, is

\begin{equation}
\text{prob}\bigg(X^{\ast} - \frac{\delta X}{2} \leq X < X^{\ast} + \frac{\delta X}{2}|I\bigg) \approx \text{prob}(X = X^{\ast} |I)\delta X,
\label{1dpdf}
\end{equation}
where the equality becomes exact in the limit $\delta X \rightarrow 0$.
Now, we want to express $p(X|I)$ as a function of $p(Y|I)$. We can do so if $X$ and $Y$ are (monotonically) related through $Y=f(X)$, then $f(X)$ will map $X^{\ast}$ to $Y^{\ast}$ and $\delta X$ to $\delta Y$. If the range of $Y$, which spans $Y^{\ast} \pm \delta Y/2$, is equivalent to the range of $X$, then the area under the pdf $p(Y|I)$ should equal the probability expressed by eq. (\ref{1dpdf}). This requires that,

\begin{equation}
\text{prob}(X = X^{\ast} |I)\delta X = \text{prob}(Y = Y^{\ast} |I)\delta Y,
\label{probXequalsprobY}
\end{equation}
This should be true for any value of X and Y, so we obtain the expression,
\begin{equation}
\text{prob}(X|I)= \text{prob}(Y|I) \times \bigg| \frac{\text{d} Y}{\text{d} X} \bigg|,
\label{jacobian}
\end{equation}
where the term in the modulus brackets is the Jacobian and it is the absolute value of the derivatives which express a ratio of lengths whether the variations of X and Y are positive or negative. 
In the multidimensional case, we can extend what we obtained in eq. (\ref{probXequalsprobY}) for $M$ parameters $\{X_{j},Y_{j}\}$,
\begin{equation}
\text{prob}(\{X_{j}\}|I)\delta X_{1} \delta X_{2}...\delta X_{M} = \text{prob}(\{Y_{j}\}|I) \delta^{M} \text{Vol}(\{Y_{j}\}),
\label{multijacobian}
\end{equation}
where the $M$-dimensional hypercube formed by $\{\delta X_{j}\}$ in the X-space, maps to a $M$-dimensional volume formed by $\{\delta Y_{j}\}$ in the Y-space throught the expression,
\begin{equation}
\delta^{M}\text{Vol}(\{Y_{j}\}) = \bigg| \frac{\partial(Y_{1},Y_{2},...,Y_{M})}{\partial(X_{1},X_{2},...,X_{M})} \bigg| \delta X_{1} \delta X_{2}...\delta X_{M},
\label{multijacobian2}
\end{equation}
where the quantity in the modulus is the multidimensional Jacobian and is the determinant of the $M \times M$ matrix of the partial derivatives $\partial Y_{i}/\partial X_{j} $. The final expression for the M-dimensional multivariate transformation is,
\begin{equation}
\text{prob}(\{X_{j}\}|I) = \text{prob}(\{Y_{j}\}|I) \times \bigg| \frac{\partial(Y_{1},Y_{2},...,Y_{M})}{\partial(X_{1},X_{2},...,X_{M})} \bigg|,
\label{multijacobian2}
\end{equation}
In our study we are interested in computing $p(m_1,m_2)$, $p(\mathcal{M},q)$ and $p(\text{M}_{total},q)$. We map $p(\mathcal{M},q)$ from $p(m_1,m_2)$ using the expression,

\begin{equation}
p(m_1,m_2)\bigg|\frac{\partial(m_1,m_2)}{\partial(\mathcal{M},q)}\bigg|=p(\mathcal{M},q),
\label{eq:m1m2tochirpMq}
\end{equation}
where
\begin{equation}
\begin{split}
\bigg| \frac{\partial(m_1,m_2)}{\partial(\mathcal{M},q)} \bigg| &=
\begin{vmatrix}
     \frac{\partial m_1}{\partial\mathcal{M}} &
     \frac{\partial m_1}{\partial q} \\ \\
     \frac{\partial m_2}{\partial \mathcal{M}} & 
     \frac{\partial m_2}{\partial q}   
\end{vmatrix} \\
&= \frac{1}{5} \mathcal{M}\bigg[ \bigg( \frac{1}{q^{2}}+\frac{1}{q^{3}} \bigg)^{\frac{1}{5}} \frac{3 q^{2} + 2 q}{(q^{3} + q^{2})^{\frac{4}{5}}} + (q^{3} + q^{2})^{\frac{1}{5}} \frac{(\frac{2}{q^{3}} + \frac{3}{q^{4}})}{(\frac{1}{q^{2}} + \frac{1}{q^{3}})^{\frac{4}{5}}}\bigg].
\label{eq:jacobianm1m2chirpMq}
\end{split}
\end{equation}
We follow the same procedure to map $p(m_1,m_2)$ to $p(M_{total},q)$,
\begin{equation}
p(m_1,m_2)\bigg|\frac{\partial(m_1,m_2)}{\partial(\text{M}_{total},q)}\bigg|=p(\text{M}_{total},q),
\label{eq:m1m2toMtotalq}
\end{equation}

where

\begin{equation}
\begin{split}
\bigg| \frac{\partial(m_1,m_2)}{\partial(\text{M}_{total},q)} \bigg| &=
\begin{vmatrix}
     \frac{\partial m_1}{\partial\text{M}_{total}} &
     \frac{\partial m_1}{\partial q} \\ \\
     \frac{\partial m_2}{\partial \text{M}_{total}} & 
     \frac{\partial m_2}{\partial q}   
\end{vmatrix} \\
&= \frac{\text{M}_{total}}{1 + q} \bigg[ \frac{1}{1 + q} - \frac{q}{(1 + q)^{2}} \bigg] + \frac{1}{1 + \frac{1}{q}} \bigg[ \frac{\text{M}_{total}}{(1 + q)^{2}} \bigg]
\label{eq:jacobianm1m2toMtotalq}
\end{split}
\end{equation}


\subsection*{Simulations}

This work consists in analysing the uncertainties associated to the observations of MBHBs with space-based observatory LISA. The parameters that we are going to study are the component masses $m_1$ and $m_2$, chirp mass $\mathcal{M}$, and mass ratio $q$. Our analyses rely on the computation of the Fisher matrices for a rapid statistical inference, rather than a full Bayesian parameter estimation, which is computational more expensive. In this process, we use the Gaussian approximation which allows us to assume the likelihoods, so as the posteriors, to have Gaussian behaviour. The Gaussian pdf for the component masses $m_1$ and $m_2$ is,

\begin{equation}
p(m_1,m_2)= \frac{1}{{(2\pi)}^{\frac{1}{2}}\big|\Sigma\big|^{\frac{1}{2}}} \mathrm{exp} \Big[-\frac{1}{2}\bigg( 
\begin{bmatrix}     
     m_1 \\
     m_2 
\end{bmatrix}
-
\begin{bmatrix}
     m^{inj}_{1} \\
     m^{inj}_{2} 
\end{bmatrix}
\bigg)^{T} \Sigma^{-1} 
\bigg( 
\begin{bmatrix}     
     m_1 \\
     m_2 
\end{bmatrix}
-
\begin{bmatrix}
     m^{inj}_{1} \\
     m^{inj}_{2} 
\end{bmatrix}
\bigg)
\Big],
\label{eq:gaussianm1m2}
\end{equation}
where $m_1$ and $m_2$ are the component masses, $\Sigma$ is the variance-covariance matrix and $m^{inj}_1$ and $m^{inj}_{2}$ are the true injected values for the component masses. We assume working in the high SNR limit, so the variance-covariance matrix is the inverse of the Fisher matrix as explained in section X. We obtain $p(\mathcal{M},q)$ by using eqs. (\ref{eq:m1m2tochirpMq}), (\ref{eq:jacobianm1m2chirpMq}) and (\ref{eq:gaussianm1m2}), and $p(M_{total},q)$ by using eqs.(\ref{eq:m1m2toMtotalq}), (\ref{eq:jacobianm1m2toMtotalq}) and (\ref{eq:gaussianm1m2}).

Of interest for us is the study of the correlations between parameters, i.e. correlations between component masses, chirp mass and mass ratio, and total mass and mass ratio. Also, some other global properties of the set of observations such as the general trend of the PDF contour plots slopes, which lets us know how the correlations between parameters evolve with chirp mass, total mass, and SNR. The slopes of the 2D PDF contour plots each observation are calculated considering the eigenvectors of the covariance matrix of each observation via the expression:

The slope is obtained from the ratio of the eigenvectors associated to the principal components of the contour plot

One important feature that we are going to delve into is the slope of these contour plots and the global behaviour these have across chirp mass. For each PDF contour plot, the slope is calculated from the quotient of the eigenvector of the major principal component axis and the eigenvector of the minor principal component axis of the contour plot, this is,

\begin{equation}
\text{slope} \ p(\cdot| \cdot) = \frac{x}{y},
\label{eq:slope}
\end{equation}

where $x$ and $y$ are the eigenvectors of the major and minor principal component axes respectively, and satisfy the characteristic equation,
\begin{equation}
\begin{bmatrix}
     A & C \\
     C & B        
\end{bmatrix} 
\begin{bmatrix}
     x \\
     y         
\end{bmatrix} = \lambda \ 
\begin{bmatrix}
     x \\
     y        
\end{bmatrix},
\label{eq:char}
\end{equation}
where $\left[ \begin{smallmatrix} A & C \\ C & B \end{smallmatrix} \right]$ is the two-parameter Fisher matrix, and $\lambda$ are the eigenvalues associated to the eigenvectors $x$ and $y$. Figures \ref{slopespm1m2}, \ref{slopespMchirpq} and \ref{slopespMtotalq} show plots of the slopes of $p(m_1,m_2)$, $p(\mathcal{M},q)$, and $p(M_{total},q)$ versus $\mathcal{M}$, respectively, for the 40 observations. In Figure \ref{slopespm1m2}, we can see the constant trend the slope of $p(m_1,m_2)$ versus $\mathcal{M}$ follows for $q=0.50$ and $0.75$, not the same for $q=0.25$ which shows some scatter. This scatter reflects a behaviour of the slopes, and consequently the correlations, where they go from negative from positive. To obtain the slopes of $p(\mathcal{M},q)$ we used a different approach: Since we need the Since we obtained $p(\mathcal{M},q)$ from $p(m1,m2)$ with a Jacobian transformation, eq.\ref{eq:jacobian}, we don't have it's covariance matrix to compute the eigenvectors. We obtain the covariance matrix of $p(\mathcal{M},q)$ from the covariance matrix of $p(m_1,m_2)$ by means of the transformation,
\begin{equation}
\mathbf{\Sigma}_{\mathcal{M}q} = \mathbf{J} \; \mathbf{\Sigma}_{m_{1} m_{2}} \; \mathbf{J}^{T},
\label{eq:transfmchirpq_m1m2}
\end{equation}
where $\mathbf{\Sigma}_{\mathcal{M}q}$ is the covariance matrix of $p(\mathcal{M},q)$, 

\begin{equation}
\mathbf{\Sigma}_{\mathcal{M}q} = 
\begin{bmatrix}
     \sigma^{2}_\mathcal{M} &
     \sigma_{\mathcal{M}q} \\ \\
     \sigma_{\mathcal{M}q} & 
     \sigma^{2}_{q}     
\end{bmatrix},
\label{eq:covmchirpq}
\end{equation}
and $\mathbf{\Sigma}_{m_{1} m_{2}}$ is the covariance matrix of $(m_1,m_2)$,
\begin{equation}
\mathbf{\Sigma}_{m_1 m_2} = 
\begin{bmatrix}
     \sigma^{2}_{m_1} &
     \sigma_{m_{1} m_{2}} \\ \\
     \sigma_{m_{1} m_{2}} & 
     \sigma^{2}_{m_2}     
\end{bmatrix},
\label{covm1m2}
\end{equation}
and $\mathbf{J}$ is the Jacobian matrix,
\begin{equation}
\mathbf{J} =
\begin{bmatrix}
     \frac{\partial \mathcal{M}}{\partial m_{1}} &
     \frac{\partial \mathcal{M}}{\partial m_{2}} \\ \\
     \frac{\partial q}{\partial m_{1}} & 
     \frac{\partial q}{\partial m_{2}}.     
\end{bmatrix},
\label{jacobianmatrix}
\end{equation}
We double checked eq. by computing $p(\mathcal{M},q)$ assuming Gaussian behaviour with covariance matrix

TEST: Added this line

\subsection*{Results}



We performed 40 simulated LISA observations of MBHBs. We specify a chirp mass $\mathcal{M}$ in the range of $(3\times10^{5},3.5\times10^{5})$ and $q=\{0.25,0.50,0.75\}$. These values were chosen so that they cover the typical total mass values for MBHBs in the range of $10^{4}$ to $10^{7}$. With these values, we obtain the component masses $m_1$ and $m_2$ that we inject into GWFish. The rest of the injected parameters, i.e, redshift, luminosity distance, inclination $\theta_{JN}$, right ascension $\alpha$, declination $\delta$ and polarisation angle are fixed to the predetermined GWFish values and are consistent with (reference).
Figures \ref{fig:PDFsq025}, \ref{fig:PDFsq050} and \ref{fig:PDFsq075} show PDFs contour plots for $p(m_1,m_2)$, $p(\mathcal{M},q)$ and $p(M_{total},q)$ for $q=0.25,0.50$, and  $0.75$ respectively. In Figure \ref{fig:pm1m2q025} shows a $p(m_1,m_2)$ with negative correlation. This is because since the total mass is the observed mass parameter from a detection, $m_2$ needs to decrease as $m_1$ increases so that the observed total mass remains constant. On the other hand, Figure \ref{fig:pMchirpqq025} shows positive correlation in which $\mathcal{M}$ increases with $q$. This has to do with regions of the sensitivity curve on which the observed signal lays. 

\begin{figure}[ht]
    \centering
    % First subfigure
    \begin{subfigure}[b]{0.45\textwidth}
        \centering
        \includegraphics[width=\textwidth]{figs_q025/pm1m2_0_m1_720700.0_m2_180200.0.pdf}
        \caption{$p(m_1,m_2)$}
        \label{fig:pm1m2q025}
    \end{subfigure}
    \hfill
    % Second subfigure
    \begin{subfigure}[b]{0.45\textwidth}
        \centering
        \includegraphics[width=\textwidth]{figs_q025/pchirpMq_0_Mc_300033.85897310276_q_0.2500346884972943.pdf}
        \caption{$p(\mathcal{M},q)$}
        \label{fig:pMchirpqq025}
    \end{subfigure}
    % Third subfigure
    \begin{subfigure}[b]{0.45\textwidth}
        \centering
        \includegraphics[width=\textwidth]{figs_q025/ptotalMq_0.pdf}
        \caption{$p(M_{total},q)$}
        \label{fig:pMtotalqq025}
    \end{subfigure}
    \hfill    
    \caption{PDFs for $q=0.25$}
    \label{fig:PDFsq025}
\end{figure}

\begin{figure}[ht]
    \centering
    % First subfigure
    \begin{subfigure}[b]{0.45\textwidth}
        \centering
        \includegraphics[width=\textwidth]{figs_q050/pm1m2_0.pdf}
        \caption{$p(m_1,m_2)$}
        \label{fig:pm1m2q050}
    \end{subfigure}
    \hfill
    % Second subfigure
    \begin{subfigure}[b]{0.45\textwidth}
        \centering
        \includegraphics[width=\textwidth]{figs_q050/pchirpMq_0.pdf}
        \caption{$p(\mathcal{M},q)$}
        \label{fig:pMchirpqq050}
    \end{subfigure}
    % Third subfigure
    \begin{subfigure}[b]{0.45\textwidth}
        \centering
        \includegraphics[width=\textwidth]{figs_q050/ptotalMq_0.pdf}
        \caption{$p(M_{total},q)$}
        \label{fig:pMtotalqq050}
    \end{subfigure}
    \hfill    
    \caption{PDFs for $q=0.50$}
    \label{fig:PDFsq050}
\end{figure}

\begin{figure}[ht]
    \centering
    % First subfigure
    \begin{subfigure}[b]{0.45\textwidth}
        \centering
        \includegraphics[width=\textwidth]{figs_q075/pm1m2_0.pdf}
        \caption{$p(m_1,m_2)$}
        \label{fig:pm1m2q075}
    \end{subfigure}
    \hfill
    % Second subfigure
    \begin{subfigure}[b]{0.45\textwidth}
        \centering
        \includegraphics[width=\textwidth]{figs_q075/pchirpMq_0.pdf}
        \caption{$p(\mathcal{M},q)$}
        \label{fig:pMchirpqq075}
    \end{subfigure}
    % Third subfigure
    \begin{subfigure}[b]{0.45\textwidth}
        \centering
        \includegraphics[width=\textwidth]{figs_q075/ptotalMq_0.pdf}
        \caption{$p(M_{total},q)$}
        \label{fig:pMtotalqq075}
    \end{subfigure}
    \hfill    
    \caption{PDFs for $q=0.75$}
    \label{fig:PDFsq075}
\end{figure}
\hfill \break
\clearpage

\begin{figure}[htbp]
\includegraphics[width=.8\textwidth]{figs_slopes/chirpM_slope_m1m2_qs.pdf}
\caption{Plot of slopes of $p(m_1,m_2)$ versus $\mathcal{M}$ for the 40 MBHB observations.}
\label{slopespm1m2}
\end{figure}

\begin{figure}[htbp]
\includegraphics[width=.8\textwidth]{figs_slopes/chirpM_slope_chirpMq_qs.pdf}
\caption{Plot of slopes of $p(\mathcal{M},q)$ versus $\mathcal{M}$ for the 40 MBHB observations.}
\label{slopespMchirpq}
\end{figure}

\begin{figure}[htbp]
\includegraphics[width=.8\textwidth]{figs_slopes/chirpM_slope_totalMq_qs.pdf}
\caption{Plot of slopes of $p(M_{total},q)$ versus $\mathcal{M}$ for the 40 MBHB observations.}
\label{slopespMtotalq}
\end{figure}

It is of high interest to compare our results with the ones in the literature. We performed a computation of $p(m_1,m_2)$ and compared our results with the results of Marsat et al. (2020). The used parameters and their values are shown in Table \ref{t:MBD}. We use source-frame masses for the injection in GWFish. For the sky localisation parameters, Marsat et al. take these values in ecliptic coordinates, ecliptic longitude $\lambda$ and latitude $\beta$, with origin in the Solar System Barycenter (SSB). We converted these into equatorial coordinates in Earth's frame, Right Ascension $\alpha$ and Declination $\delta$, to be introduced in GWFish.

In order to have a consistent setting for the comparison, we implement the PSD curve prescription provided in Marsat et al (2020),

\begin{equation}
S^{a,e,t}_c(f) = f S^{a,e,t}_h(f),
\label{charPSD}
\end{equation}
where $S^{a,e,t}_c(f)$ is the characteristic noise PSD and $f$ is the frequency. $S^{a,e,t}_h(f)$ is,

\begin{equation}
S^{a,e,t}_h(f) = \frac{S^{a,e,t}_n(f)}{(6\pi fL)^{2}},
\label{strainnoisePSD}
\end{equation}
where $S^{a,e,t}_n(f)$ is the reduced PSD for the $a$, $e$ and $t$ TDI observables,
\begin{subequations}
\begin{align}
S^{a}_n = S^{e}_n 
&= \frac{P^{a,e}_n}{\mathcal{R}(f)},
\label{redPSDa} \\
S^{t}_n 
&= \frac{P^{t}_n}{\mathcal{R}(f)},
\label{redPSDb}
\end{align}
\end{subequations}
and, 
\begin{subequations}
\begin{align}
P^{a}_n = P^{e}_n 
&= 2(3+2\text{cos}(2\pi fL)+ \text{cos}(4\pi fL))S^{\text{pm}}(f)+(2+\text{cos}(2\pi fL))S^{\text{op}}(f)
\label{redPSDrespa} \\
P^{t}_n 
&= 4\text{sin}^{2}(2\pi fL)S^{\text{pm}}(f)+S^{\text{op}}(f),
\label{redPSDrespb}
\end{align}
\end{subequations}

and $S^{\text{op}}$ and $S^{\text{pm}}$ are the optical noise PSD and test-mass noise PSD respectively. Figure \ref{psdcurves} shows the PSD curve used in our study. We have plotted the PSD curve used in Marsat et al paper in Figure X as well, noticing there is a slight difference between the one computed with eq. X and the one shown in Marsat's paper.

\begin{figure}[htbp]
\includegraphics[width=.8\textwidth]{Figures/Sc_Curves_Marsat.pdf}
\caption{PSD curve computed with eq. \ref{redPSDa}}
\label{psdcurves}
\end{figure}

To reproduce the results of Marsat from our simulation, we need to consider the LISA detector position at the time the detection is made, the same detector position as in Marsat et al. Since the paper does not explicitly show the detector position, we follow another way around: We set out the same SNR as in Marsat by running a simulation to compute the SNR for a range of geocentric times, and track the geocentric time which yields the SNR of interest. Figure \ref{snrgeocentric} shows the SNR versus geocentric time.
\begin{figure}[htbp]
\includegraphics[width=.8\textwidth]{Figures/SNR_GeocentTime.png}
\caption{SNR versus geocentric time}
\label{snrgeocentric}
\end{figure}


\end{document}
